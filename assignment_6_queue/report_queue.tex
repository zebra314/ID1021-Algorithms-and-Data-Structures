\documentclass[a4paper,11pt]{article}

\usepackage[utf8]{inputenc}

\usepackage{graphicx}
\usepackage{caption}
\usepackage{subcaption}
\usepackage{float}
\usepackage{amsmath}
\usepackage[colorlinks=true, linkcolor=blue, urlcolor=blue]{hyperref}

\usepackage{pgfplots}
\pgfplotsset{compat=1.18} 

\usepackage{minted}
\usepackage{pgfplotstable}

\begin{document}

\title{
  \textbf{Queues in C}
}
\author{Ying Pei Lin}
\date{Fall 2024}

\maketitle

\section*{Implementing a queue by linked list}

To implement a queue in C with linked list, we have to track the first cell and the next
cell of each cell. 

To initialize a queue, we allocate memory for the queue and set the first cell to {\tt NULL}.

\begin{minted}{c}
queue *create_queue() {
  queue *q = (queue*)malloc(sizeof(queue));
  q->first = NULL;
  return q;
}
\end{minted}

To check if the queue is empty, we check if the first cell is {\tt NULL}.

\begin{minted}{c}
int empty(queue *q) {
  return q->first == NULL;
}
\end{minted}

To add an element to the queue, we allocate memory for the new node, set the value,
and find the last element in the queue. If the last element is not {\tt NULL}, which means
the queue is not empty, we set the next of the last element to the new node. Otherwise,
the new node is the first element in the queue.

\begin{minted}{c}
void enque(queue* q, int v) {
  // Init the new element
  node *nd = (node*)malloc(sizeof(node));
  nd->value = v;
  nd->next = NULL;

  // Find the last element in the queue
  node *prv = NULL;
  node *nxt = q->first;
  while (nxt != NULL) {
    prv = nxt;
    nxt = nxt->next;
  }

  // Check if the new one is the first element in the queue
  if (prv != NULL) {
    prv->next = nd;
  } else {
    q->first = nd;
  }
}
\end{minted}

To remove an element from the queue, we free the first element and set the 
next element as the new first element. Note that if the queue is empty,
the first element is {\tt NULL} and the function will return 0, and therefore
the queue can only store integers other than 0.

\begin{minted}{c}
int dequeue(queue *q) {
  int res = 0;
  if (q->first != NULL) {
    node* temp = q->first;
    res = temp->value;
    q->first = q->first->next;
    free(temp);  
  }
  return res;
}
\end{minted}

\section*{Improving the queue}

In the previous implementation, we have to iterate through the queue to find the 
last element to add the new element next to it. This operation is $O(n)$, which is
not efficient. By keeping track of the last element as we track the first element,
we can add the new element in $O(1)$ time.

The optimized implementation is as follows:

\begin{minted}{c}
void enque(queue* q, int v) {
  // Init the new element
  node *nd = (node*)malloc(sizeof(node));
  nd->value = v;
  nd->next = NULL;

  // Check if the new one is the first element in the queue
  if (q->last != NULL) {
    q->last->next = nd;
  } else {
    q->first = nd;
  }
  q->last = nd; // Update the last element
}
\end{minted}


\end{document}
