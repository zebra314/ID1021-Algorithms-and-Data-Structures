\documentclass[a4paper,11pt]{article}

\usepackage[utf8]{inputenc}

\usepackage{graphicx}
\usepackage{caption}
\usepackage{subcaption}
\usepackage{float}
\usepackage{amsmath}
\usepackage[colorlinks=true, linkcolor=blue, urlcolor=blue]{hyperref}

\usepackage{pgfplots}
\pgfplotsset{compat=1.18} 

\usepackage{minted}
\usepackage{pgfplotstable}

\begin{document}

\title{
  \textbf{Sorting an array in C}
}
\author{Ying Pei Lin}
\date{Fall 2024}

\maketitle

\section*{Selection Sort}

From the start of the array, the algorithm searches for the smallest element in 
the unsorted part of the array(the right side of the current element) and swaps
it with the current element, then moves to the next element. If there are n elements
in the array, the algorithm will perform $(n-i)$ comparisons for the $i$th element, 
and the total number of comparisons is:

\begin{equation}
  n + (n-1) + (n-2) + \ldots + 1 = \frac{n(n+1)}{2}
\end{equation}

Therefore, the time complexity of the selection sort is $O(n^2)$.
The following code snippet shows the implementation of the selection sort algorithm
and the result of sorting array with different sizes is shown in Figure \ref{fig:selection-sort}.

\begin{minted}{c}
void selection_sort(int *array, int length) {
  for (int i = 0; i < length - 1; i++) {
    int min_index = i;
    for (int j = i + 1; j < length; j++) {
      if (array[j] < array[min_index]) {
        min_index = j;
      }
    }
    int temp = array[min_index];
    array[min_index] = array[i];
    array[i] = temp;
  }
}
\end{minted}

\begin{figure}[H]
  \centering
  \begin{tikzpicture}
  \begin{axis}[
    xlabel={Array Size (per M element)},
    ylabel={Time (ms)},
    xmode=log,
    log basis x=10,
    ymode=log,
    log basis y=10,
    ymajorgrids=true,
    grid style=dashed,
    width=12cm, height=6cm,
    % legend pos=outer north east,
    scaled y ticks=false,
    scaled x ticks=false,
    % y filter/.code={\pgfmathparse{#1/1000}\pgfmathresult},
    % x filter/.code={\pgfmathparse{#1/1000000}\pgfmathresult},
    ]
  \addplot[color=blue,] table {./data/selection_sort.dat};
  \end{axis}

  \end{tikzpicture}
  \caption{Selection Sort with different array sizes}
  \label{fig:selection-sort}
\end{figure}

\section*{Insertion Sort}

The insertion sort algorithm works by finding the correct position at the left 
side of the current element. If there are n elements in the array, the algorithm
will perform $(n-i)$ comparisons for the $i$th element, and the total number of
comparisons is the same as the selection sort algorithm, which is $\frac{n(n+1)}{2}$.
Therefore, the time complexity of the insertion sort is also $O(n^2)$.

The following code snippet implements the insertion sort algorithm and the result
is shown in Figure \ref{fig:insertion-sort}.

\begin{minted}{c}
void insertion_sort(int *array, int length) {
  int i, j, key;
  for (i = 1; i < length; i++) {
    key = array[i];
    j = i - 1;
    while (j >= 0 && array[j] > key) {
      array[j + 1] = array[j];
      j = j - 1;
    }
    array[j + 1] = key;
  }
}
\end{minted}
  
\begin{figure}[H]
  \centering
  \begin{tikzpicture}
  \begin{axis}[
    xlabel={Array Size (per M element)},
    ylabel={Time (ms)},
    xmode=log,
    log basis x=10,
    ymode=log,
    log basis y=10,
    ymajorgrids=true,
    grid style=dashed,
    width=12cm, height=6cm,
    % legend pos=outer north east,
    scaled y ticks=false,
    scaled x ticks=false,
    % y filter/.code={\pgfmathparse{#1/1000}\pgfmathresult},
    % x filter/.code={\pgfmathparse{#1/1000000}\pgfmathresult},
    ]
  \addplot[color=blue,] table {./data/insertion_sort.dat};
  \end{axis}

  \end{tikzpicture}
  \caption{Insertion Sort with different array sizes}
  \label{fig:insertion-sort}
\end{figure}

Compare to the selection sort, the insertion sort algorithm is more efficient
when the array is partially sorted. Additionally, the insertion sort is considered
more stable than the selection sort as it does not change the order of the elements
that has the right order, while the selection sort does sometimes.

\section*{Merge Sort}

The merge sort alogrithm divides the array into two halves recursively, then merge
the two halves in a sorted order. For a array with n elements, the algorithm will
divides the array $\log_2 n$ times, until each subarray has only one element.
Then, the algorithm will merge the subarrays, which takes $O(n)$ time. Therefore,
the time complexity of the merge sort is $O(n \log n)$.

The implementation of the merge sort alogrithm is a little too long to show here,
I uploaded the code to the \href{https://github.com/zebra314/ID1021-Algorithms-and-Data-Structures/blob/main/assignment\_4/merge\_sort.c}{file in github repository}.
The result of sorting array with different sizes is shown in Figure \ref{fig:merge-sort}.
We can see that the Time per element($T/n$) is proportional to the size of the array in a log scale($\log_{10}n$),
which means $T \propto n \log n$.

\begin{figure}[H]
  \centering
  \begin{tikzpicture}
  \begin{axis}[
    xlabel={Array Size (per M element, log scale)},
    ylabel={Time per element ($T/n$, ms)},
    xmode=log,
    log basis x=10,
    ymajorgrids=true,
    grid style=dashed,
    width=12cm, height=6cm,
    scaled y ticks=false,
    scaled x ticks=false,
    % Customize the tick labels (optional)
    yticklabel style={/pgf/number format/fixed, /pgf/number format/precision=3},
    ]
    
  % Apply transformation y -> (y/x) * constant (where constant is chosen as 1000)
  \addplot[color=blue] table[y expr=(\thisrowno{1}/\thisrowno{0})*1000] {./data/merge_sort.dat};
  
  \end{axis}
  \end{tikzpicture}
  \caption{Merge Sort with different array sizes (normalized as $T/n$)}
  \label{fig:merge-sort}
\end{figure}
  
\section*{Quick Sort}

The quick sort algorithm start by selecting a pivot element, then partition the array
into two parts recursively, one with elements less than the pivot and the other with elements greater
than the pivot. The time complexity of the quick sort is the same as the merge sort, which is $O(n \log n)$.

The code snippet shows the implementation of the quick sort algorithm and the result of sorting
array with different sizes is shown in Figure \ref{fig:quick-sort}.

\begin{minted}{c}
void quick_sort(Array *array, int start, int end) {
  if (start < end) {
    int pivot = partition(array, start, end);
    quick_sort(array, start, pivot - 1);
    quick_sort(array, pivot + 1, end);
  }
}

int partition(Array *array, int start, int end) {
  int pivot = array->array[end];
  int i = start - 1;
  for (int j = start; j < end; j++) {
    if (array->array[j] < pivot) {
      i++;
      swap(&array->array[i], &array->array[j]);
    }
  }
  swap(&array->array[i + 1], &array->array[end]);
  return i + 1;
}
\end{minted}

\begin{figure}[H]
  \centering
  \begin{tikzpicture}
  \begin{axis}[
    xlabel={Array Size (per M element, log scale)},
    ylabel={Time per element ($T/n$, ms)},
    xmode=log,
    log basis x=10,
    ymajorgrids=true,
    grid style=dashed,
    width=12cm, height=6cm,
    scaled y ticks=false,
    scaled x ticks=false,
    % Customize the tick labels (optional)
    yticklabel style={/pgf/number format/fixed, /pgf/number format/precision=3},
    ]
    
  % Apply transformation y -> (y/x) * constant (where constant is chosen as 1000)
  \addplot[color=blue] table[y expr=(\thisrowno{1}/\thisrowno{0})*1000] {./data/quick_sort.dat};
  
  \end{axis}
  \end{tikzpicture}
  \caption{Quick Sort with different array sizes (normalized as $T/n$)}
  \label{fig:quick-sort}
\end{figure}

\section*{Optimization of Merge Sort}

After implementing the merge sort algorithm, I found that it copies the array reapeatedly
when merging the subarrays. By following the suggestion and the resource online, I modified
the way of merging the subarrays by using the same array($array aux$) to store the sorted elements.

Originally, there are two arrays, right and left, to store the elements of the subarrays, then will
be merged into another new array. The new implementation uses the same array to store the sorted elements
and merge the subarrays in the same place.

The code snippet below shows the implementation of the optimized merge sort algorithm and the result
of sorting array with different sizes is shown in Figure \ref{fig:merge-sort-optimized}.

\begin{minted}{c}
Array merge_sort_v2(Array array) {
  int* aux = (int*)malloc(array.length * sizeof(int));
  merge_array_v2(&array, aux, 0, array.length - 1);
  free(aux);
  return array;
}

void merge_array_v2(Array* array, int* aux, int left, int right) {
  // If there is only one element in the array, sorted
  if (right <= left) return;

  int mid = left + (right - left) / 2;
  merge_array_v2(array, aux, left, mid);
  merge_array_v2(array, aux, mid + 1, right);

  // Copy the current range to the auxiliary array
  // This prevent us from copying the array multiple times
  for (int k = left; k <= right; k++) {
    aux[k] = array->array[k];
  }

  // Merge Step
  int i = left, j = mid + 1;
  for (int k = left; k <= right; k++) {

    // If the elements from one side have been merged,
    // copy the remaining elements from the other side
    if (i > mid) array->array[k] = aux[j++];
    else if (j > right) array->array[k] = aux[i++];

    // Compare the elements from the left and right array
    else if (aux[j] < aux[i]) array->array[k] = aux[j++];
    else array->array[k] = aux[i++];
  }
}
\end{minted}

\begin{figure}[H]
  \centering
  \begin{tikzpicture}
  \begin{axis}[
    xlabel={Array Size (per M element, log scale)},
    ylabel={Time per element ($T/n$, ms)},
    xmode=log,
    log basis x=10,
    ymajorgrids=true,
    grid style=dashed,
    width=12cm, height=6cm,
    scaled y ticks=false,
    scaled x ticks=false,
    legend pos=outer north east,
    % Customize the tick labels (optional)
    yticklabel style={/pgf/number format/fixed, /pgf/number format/precision=3},
    ]
  \legend{Merge Sort, Merge Sort Optimized}
  % Apply transformation y -> (y/x) * constant (where constant is chosen as 1000)
  \addplot[color=blue] table[y expr=(\thisrowno{1}/\thisrowno{0})*1000] {./data/merge_sort.dat};
  \addplot[color=red] table[y expr=(\thisrowno{1}/\thisrowno{0})*1000] {./data/merge_sort_opt.dat};
  \end{axis}
  \end{tikzpicture}
  \caption{Two types of Merge Sort (normalized as $T/n$)}
  \label{fig:merge-sort-optimized}
\end{figure}

\end{document}
