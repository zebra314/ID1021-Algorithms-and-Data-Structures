\documentclass[a4paper,11pt]{article}

\usepackage[utf8]{inputenc}

\usepackage{graphicx}
\usepackage{caption}
\usepackage{subcaption}
\usepackage{float}
\usepackage{amsmath}

\usepackage{pgfplots}
\pgfplotsset{compat=1.18} 

\usepackage{minted}
\usepackage{pgfplotstable}

\begin{document}

\title{
  \textbf{Searching in a sorted array in C}
}
\author{Ying Pei Lin}
\date{Fall 2024}

\maketitle

\section*{Unsorted Search}
The following code search a value in an unsorted array. To prevent the influence of overhead,
I made a little change to how the test data is generated. Originally, the test data is generated separately
in each iteration. Now, the test data is generated once and the clock get time function is called before and after
the for loop that iterates through different array sizes.

\begin{minted}{c}
int main() {
  srand(time(NULL));
  int loop = 1000;
  int array_size[] = {10, 100, 1000, 10000, 100000, 1000000};
  int n = sizeof(array_size) / sizeof(array_size[0]);
  TestData test_data = get_test_data(loop, array_size, n);
  for(int i = 0; i < n; i++) {
    clock_gettime(CLOCK_MONOTONIC, &t_start);
    for(int j = 0; j < loop; j++) {
      search(test_data.array_list[i][j], array_size[i], test_data.key_list[i][j]);
    }
    clock_gettime(CLOCK_MONOTONIC, &t_stop);
    printf("%d %0.2ld\n", array_size[i], nano_seconds(&t_start, &t_stop)/loop);
  }
}
\end{minted}

The result is shown in the Figure \ref{fig:unsorted_search}. It shows that the time taken to 
search linearly in an unsorted array is linearly proportional to the size of the array.

\begin{figure}[H]
  \centering
  \begin{tikzpicture}
  \begin{axis}[
    xlabel={Array Size (per M element)},
    ylabel={Time (ms)},
    ymajorgrids=true,
    grid style=dashed,
    width=12cm, height=6cm,
    legend pos=outer north east,
    scaled y ticks=false,
    scaled x ticks=false,
    y filter/.code={\pgfmathparse{#1/1000000}\pgfmathresult},
    x filter/.code={\pgfmathparse{#1/1000000}\pgfmathresult},
    ]
  \addplot[color=blue,] table {data/unsorted_search.dat};
  \end{axis}

  \end{tikzpicture}
  \caption{Time taken to search linearly in an unsorted array}
  \label{fig:unsorted_search}
\end{figure}

\section*{Binary Search}

The following code implements the binary seach algorithm. At the beginning, the {\tt first}
and {\tt last} index is set to upper and lower bound of the array.
The while loop will continue until the key is found or the first index is greater 
than the last index, which means the key is not found.
The {\tt index} is calculated by taking the average of the first and 
last index. If the value at the index is equal to the key, the function will return 
true. 

If the value at the index is less than the key and the index is less than the 
last index, which means the key is at the right side of the index.
The first index will be updated to index+1, the range of search shrinks to only
the right side of the {\tt index}.

If the value at the index is greater than the key and the index is greater than 
the first index, which means the key is at the left side of the index.
The last index will be updated to index-1, the range of search shrinks to only
the left side of the {\tt index}.

\begin{minted}{c}
bool binary_search(int array[], int length, int key) {
  int first = 0;
  int last = length-1;
  while (true) {
    int index = (first + last) / 2; // jump to the middle
    if (array[index] == key) {
      return true;
    } else if (array[index] < key && index < last) {
      first = index + 1;
    }else if (array[index] > key && index > first) {
      last = index - 1;
    } else {
      return false; // Not found
    }
  }
}
\end{minted}

The result is shown in the Figure \ref{fig:binary_search}. 
The time taken to perform binary search is proportional to the logarithm of the 
size of the array. This is because the binary search algorithm divides 
the array into half in each iteration, the number of iteration then will be 
proportional to $\log_{2}(\text{size of the array})$, namely $O(\log n)$.

\begin{figure}[H]
  \centering
  \begin{tikzpicture}
  \begin{axis}[
    xlabel={Array Size (per element)},
    ylabel={Time (ms)},
    xmode=log,
    log basis x=10,
    ymajorgrids=true,
    grid style=dashed,
    width=12cm, height=6cm,
    legend pos=outer north east,
    scaled y ticks=false,
    scaled x ticks=false,
    % y filter/.code={\pgfmathparse{#1/1000000}\pgfmathresult},
    % x filter/.code={\pgfmathparse{#1/1000000}\pgfmathresult},
    ]
  \addplot[color=blue,] table {data/binary_search.dat};
  \end{axis}

  \end{tikzpicture}
  \caption{Time taken to search using binary search}
  \label{fig:binary_search}
\end{figure}

To estimate the time it takes to search in an array of size $64M$, I first have to
find the regression line that fits the data. I use the following code to fit 
the data and find the regression function.

\begin{minted}{python}
import numpy as np
import matplotlib.pyplot as plt
data = np.loadtxt('./data/binary_search.dat')
array_size = data[:, 0]
log_array_size = np.log2(array_size)
time_ms = data[:, 1]  
a, b = np.polyfit(log_array_size, time_ms, 1)
print(f"Linear regression equation: Time = {a} * log2(ArraySize) + {b}")
\end{minted}

The result is shown in the Figure \ref{fig:regression}. The regression line is
$Time = 53.17 * \log_{2}(\text{ArraySize}) - 239.86$. The time taken to search in an array of size $64M$ is
$53.17 * \log_{2}(64M) - 239.86 = 1134.84$ ms $\approx 1100 $ ms.

\begin{figure}[H]
  \centering
  \resizebox{\linewidth}{!}{
    %% Creator: Matplotlib, PGF backend
%%
%% To include the figure in your LaTeX document, write
%%   \input{<filename>.pgf}
%%
%% Make sure the required packages are loaded in your preamble
%%   \usepackage{pgf}
%%
%% Also ensure that all the required font packages are loaded; for instance,
%% the lmodern package is sometimes necessary when using math font.
%%   \usepackage{lmodern}
%%
%% Figures using additional raster images can only be included by \input if
%% they are in the same directory as the main LaTeX file. For loading figures
%% from other directories you can use the `import` package
%%   \usepackage{import}
%%
%% and then include the figures with
%%   \import{<path to file>}{<filename>.pgf}
%%
%% Matplotlib used the following preamble
%%   \def\mathdefault#1{#1}
%%   \everymath=\expandafter{\the\everymath\displaystyle}
%%   
%%   \ifdefined\pdftexversion\else  % non-pdftex case.
%%     \usepackage{fontspec}
%%   \fi
%%   \makeatletter\@ifpackageloaded{underscore}{}{\usepackage[strings]{underscore}}\makeatother
%%
\begingroup%
\makeatletter%
\begin{pgfpicture}%
\pgfpathrectangle{\pgfpointorigin}{\pgfqpoint{6.400000in}{4.800000in}}%
\pgfusepath{use as bounding box, clip}%
\begin{pgfscope}%
\pgfsetbuttcap%
\pgfsetmiterjoin%
\definecolor{currentfill}{rgb}{1.000000,1.000000,1.000000}%
\pgfsetfillcolor{currentfill}%
\pgfsetlinewidth{0.000000pt}%
\definecolor{currentstroke}{rgb}{1.000000,1.000000,1.000000}%
\pgfsetstrokecolor{currentstroke}%
\pgfsetdash{}{0pt}%
\pgfpathmoveto{\pgfqpoint{0.000000in}{0.000000in}}%
\pgfpathlineto{\pgfqpoint{6.400000in}{0.000000in}}%
\pgfpathlineto{\pgfqpoint{6.400000in}{4.800000in}}%
\pgfpathlineto{\pgfqpoint{0.000000in}{4.800000in}}%
\pgfpathlineto{\pgfqpoint{0.000000in}{0.000000in}}%
\pgfpathclose%
\pgfusepath{fill}%
\end{pgfscope}%
\begin{pgfscope}%
\pgfsetbuttcap%
\pgfsetmiterjoin%
\definecolor{currentfill}{rgb}{1.000000,1.000000,1.000000}%
\pgfsetfillcolor{currentfill}%
\pgfsetlinewidth{0.000000pt}%
\definecolor{currentstroke}{rgb}{0.000000,0.000000,0.000000}%
\pgfsetstrokecolor{currentstroke}%
\pgfsetstrokeopacity{0.000000}%
\pgfsetdash{}{0pt}%
\pgfpathmoveto{\pgfqpoint{0.717889in}{0.564143in}}%
\pgfpathlineto{\pgfqpoint{5.276244in}{0.564143in}}%
\pgfpathlineto{\pgfqpoint{5.276244in}{4.650000in}}%
\pgfpathlineto{\pgfqpoint{0.717889in}{4.650000in}}%
\pgfpathlineto{\pgfqpoint{0.717889in}{0.564143in}}%
\pgfpathclose%
\pgfusepath{fill}%
\end{pgfscope}%
\begin{pgfscope}%
\pgfpathrectangle{\pgfqpoint{0.717889in}{0.564143in}}{\pgfqpoint{4.558356in}{4.085857in}}%
\pgfusepath{clip}%
\pgfsetbuttcap%
\pgfsetroundjoin%
\definecolor{currentfill}{rgb}{0.000000,0.000000,1.000000}%
\pgfsetfillcolor{currentfill}%
\pgfsetlinewidth{1.003750pt}%
\definecolor{currentstroke}{rgb}{0.000000,0.000000,1.000000}%
\pgfsetstrokecolor{currentstroke}%
\pgfsetdash{}{0pt}%
\pgfsys@defobject{currentmarker}{\pgfqpoint{-0.041667in}{-0.041667in}}{\pgfqpoint{0.041667in}{0.041667in}}{%
\pgfpathmoveto{\pgfqpoint{0.000000in}{-0.041667in}}%
\pgfpathcurveto{\pgfqpoint{0.011050in}{-0.041667in}}{\pgfqpoint{0.021649in}{-0.037276in}}{\pgfqpoint{0.029463in}{-0.029463in}}%
\pgfpathcurveto{\pgfqpoint{0.037276in}{-0.021649in}}{\pgfqpoint{0.041667in}{-0.011050in}}{\pgfqpoint{0.041667in}{0.000000in}}%
\pgfpathcurveto{\pgfqpoint{0.041667in}{0.011050in}}{\pgfqpoint{0.037276in}{0.021649in}}{\pgfqpoint{0.029463in}{0.029463in}}%
\pgfpathcurveto{\pgfqpoint{0.021649in}{0.037276in}}{\pgfqpoint{0.011050in}{0.041667in}}{\pgfqpoint{0.000000in}{0.041667in}}%
\pgfpathcurveto{\pgfqpoint{-0.011050in}{0.041667in}}{\pgfqpoint{-0.021649in}{0.037276in}}{\pgfqpoint{-0.029463in}{0.029463in}}%
\pgfpathcurveto{\pgfqpoint{-0.037276in}{0.021649in}}{\pgfqpoint{-0.041667in}{0.011050in}}{\pgfqpoint{-0.041667in}{0.000000in}}%
\pgfpathcurveto{\pgfqpoint{-0.041667in}{-0.011050in}}{\pgfqpoint{-0.037276in}{-0.021649in}}{\pgfqpoint{-0.029463in}{-0.029463in}}%
\pgfpathcurveto{\pgfqpoint{-0.021649in}{-0.037276in}}{\pgfqpoint{-0.011050in}{-0.041667in}}{\pgfqpoint{0.000000in}{-0.041667in}}%
\pgfpathlineto{\pgfqpoint{0.000000in}{-0.041667in}}%
\pgfpathclose%
\pgfusepath{stroke,fill}%
}%
\begin{pgfscope}%
\pgfsys@transformshift{0.925087in}{1.019442in}%
\pgfsys@useobject{currentmarker}{}%
\end{pgfscope}%
\begin{pgfscope}%
\pgfsys@transformshift{1.533939in}{1.146130in}%
\pgfsys@useobject{currentmarker}{}%
\end{pgfscope}%
\begin{pgfscope}%
\pgfsys@transformshift{2.142792in}{1.757940in}%
\pgfsys@useobject{currentmarker}{}%
\end{pgfscope}%
\begin{pgfscope}%
\pgfsys@transformshift{2.568361in}{2.153453in}%
\pgfsys@useobject{currentmarker}{}%
\end{pgfscope}%
\begin{pgfscope}%
\pgfsys@transformshift{2.751644in}{2.323401in}%
\pgfsys@useobject{currentmarker}{}%
\end{pgfscope}%
\begin{pgfscope}%
\pgfsys@transformshift{3.177214in}{2.681835in}%
\pgfsys@useobject{currentmarker}{}%
\end{pgfscope}%
\begin{pgfscope}%
\pgfsys@transformshift{3.360497in}{2.768353in}%
\pgfsys@useobject{currentmarker}{}%
\end{pgfscope}%
\begin{pgfscope}%
\pgfsys@transformshift{3.786066in}{3.528481in}%
\pgfsys@useobject{currentmarker}{}%
\end{pgfscope}%
\begin{pgfscope}%
\pgfsys@transformshift{3.969349in}{3.611910in}%
\pgfsys@useobject{currentmarker}{}%
\end{pgfscope}%
\end{pgfscope}%
\begin{pgfscope}%
\pgfsetbuttcap%
\pgfsetroundjoin%
\definecolor{currentfill}{rgb}{0.000000,0.000000,0.000000}%
\pgfsetfillcolor{currentfill}%
\pgfsetlinewidth{0.803000pt}%
\definecolor{currentstroke}{rgb}{0.000000,0.000000,0.000000}%
\pgfsetstrokecolor{currentstroke}%
\pgfsetdash{}{0pt}%
\pgfsys@defobject{currentmarker}{\pgfqpoint{0.000000in}{-0.048611in}}{\pgfqpoint{0.000000in}{0.000000in}}{%
\pgfpathmoveto{\pgfqpoint{0.000000in}{0.000000in}}%
\pgfpathlineto{\pgfqpoint{0.000000in}{-0.048611in}}%
\pgfusepath{stroke,fill}%
}%
\begin{pgfscope}%
\pgfsys@transformshift{1.232648in}{0.564143in}%
\pgfsys@useobject{currentmarker}{}%
\end{pgfscope}%
\end{pgfscope}%
\begin{pgfscope}%
\definecolor{textcolor}{rgb}{0.000000,0.000000,0.000000}%
\pgfsetstrokecolor{textcolor}%
\pgfsetfillcolor{textcolor}%
\pgftext[x=1.232648in,y=0.466921in,,top]{\color{textcolor}{\rmfamily\fontsize{10.000000}{12.000000}\selectfont\catcode`\^=\active\def^{\ifmmode\sp\else\^{}\fi}\catcode`\%=\active\def%{\%}$2^{5}$}}%
\end{pgfscope}%
\begin{pgfscope}%
\pgfsetbuttcap%
\pgfsetroundjoin%
\definecolor{currentfill}{rgb}{0.000000,0.000000,0.000000}%
\pgfsetfillcolor{currentfill}%
\pgfsetlinewidth{0.803000pt}%
\definecolor{currentstroke}{rgb}{0.000000,0.000000,0.000000}%
\pgfsetstrokecolor{currentstroke}%
\pgfsetdash{}{0pt}%
\pgfsys@defobject{currentmarker}{\pgfqpoint{0.000000in}{-0.048611in}}{\pgfqpoint{0.000000in}{0.000000in}}{%
\pgfpathmoveto{\pgfqpoint{0.000000in}{0.000000in}}%
\pgfpathlineto{\pgfqpoint{0.000000in}{-0.048611in}}%
\pgfusepath{stroke,fill}%
}%
\begin{pgfscope}%
\pgfsys@transformshift{1.782497in}{0.564143in}%
\pgfsys@useobject{currentmarker}{}%
\end{pgfscope}%
\end{pgfscope}%
\begin{pgfscope}%
\definecolor{textcolor}{rgb}{0.000000,0.000000,0.000000}%
\pgfsetstrokecolor{textcolor}%
\pgfsetfillcolor{textcolor}%
\pgftext[x=1.782497in,y=0.466921in,,top]{\color{textcolor}{\rmfamily\fontsize{10.000000}{12.000000}\selectfont\catcode`\^=\active\def^{\ifmmode\sp\else\^{}\fi}\catcode`\%=\active\def%{\%}$2^{8}$}}%
\end{pgfscope}%
\begin{pgfscope}%
\pgfsetbuttcap%
\pgfsetroundjoin%
\definecolor{currentfill}{rgb}{0.000000,0.000000,0.000000}%
\pgfsetfillcolor{currentfill}%
\pgfsetlinewidth{0.803000pt}%
\definecolor{currentstroke}{rgb}{0.000000,0.000000,0.000000}%
\pgfsetstrokecolor{currentstroke}%
\pgfsetdash{}{0pt}%
\pgfsys@defobject{currentmarker}{\pgfqpoint{0.000000in}{-0.048611in}}{\pgfqpoint{0.000000in}{0.000000in}}{%
\pgfpathmoveto{\pgfqpoint{0.000000in}{0.000000in}}%
\pgfpathlineto{\pgfqpoint{0.000000in}{-0.048611in}}%
\pgfusepath{stroke,fill}%
}%
\begin{pgfscope}%
\pgfsys@transformshift{2.332346in}{0.564143in}%
\pgfsys@useobject{currentmarker}{}%
\end{pgfscope}%
\end{pgfscope}%
\begin{pgfscope}%
\definecolor{textcolor}{rgb}{0.000000,0.000000,0.000000}%
\pgfsetstrokecolor{textcolor}%
\pgfsetfillcolor{textcolor}%
\pgftext[x=2.332346in,y=0.466921in,,top]{\color{textcolor}{\rmfamily\fontsize{10.000000}{12.000000}\selectfont\catcode`\^=\active\def^{\ifmmode\sp\else\^{}\fi}\catcode`\%=\active\def%{\%}$2^{11}$}}%
\end{pgfscope}%
\begin{pgfscope}%
\pgfsetbuttcap%
\pgfsetroundjoin%
\definecolor{currentfill}{rgb}{0.000000,0.000000,0.000000}%
\pgfsetfillcolor{currentfill}%
\pgfsetlinewidth{0.803000pt}%
\definecolor{currentstroke}{rgb}{0.000000,0.000000,0.000000}%
\pgfsetstrokecolor{currentstroke}%
\pgfsetdash{}{0pt}%
\pgfsys@defobject{currentmarker}{\pgfqpoint{0.000000in}{-0.048611in}}{\pgfqpoint{0.000000in}{0.000000in}}{%
\pgfpathmoveto{\pgfqpoint{0.000000in}{0.000000in}}%
\pgfpathlineto{\pgfqpoint{0.000000in}{-0.048611in}}%
\pgfusepath{stroke,fill}%
}%
\begin{pgfscope}%
\pgfsys@transformshift{2.882194in}{0.564143in}%
\pgfsys@useobject{currentmarker}{}%
\end{pgfscope}%
\end{pgfscope}%
\begin{pgfscope}%
\definecolor{textcolor}{rgb}{0.000000,0.000000,0.000000}%
\pgfsetstrokecolor{textcolor}%
\pgfsetfillcolor{textcolor}%
\pgftext[x=2.882194in,y=0.466921in,,top]{\color{textcolor}{\rmfamily\fontsize{10.000000}{12.000000}\selectfont\catcode`\^=\active\def^{\ifmmode\sp\else\^{}\fi}\catcode`\%=\active\def%{\%}$2^{14}$}}%
\end{pgfscope}%
\begin{pgfscope}%
\pgfsetbuttcap%
\pgfsetroundjoin%
\definecolor{currentfill}{rgb}{0.000000,0.000000,0.000000}%
\pgfsetfillcolor{currentfill}%
\pgfsetlinewidth{0.803000pt}%
\definecolor{currentstroke}{rgb}{0.000000,0.000000,0.000000}%
\pgfsetstrokecolor{currentstroke}%
\pgfsetdash{}{0pt}%
\pgfsys@defobject{currentmarker}{\pgfqpoint{0.000000in}{-0.048611in}}{\pgfqpoint{0.000000in}{0.000000in}}{%
\pgfpathmoveto{\pgfqpoint{0.000000in}{0.000000in}}%
\pgfpathlineto{\pgfqpoint{0.000000in}{-0.048611in}}%
\pgfusepath{stroke,fill}%
}%
\begin{pgfscope}%
\pgfsys@transformshift{3.432043in}{0.564143in}%
\pgfsys@useobject{currentmarker}{}%
\end{pgfscope}%
\end{pgfscope}%
\begin{pgfscope}%
\definecolor{textcolor}{rgb}{0.000000,0.000000,0.000000}%
\pgfsetstrokecolor{textcolor}%
\pgfsetfillcolor{textcolor}%
\pgftext[x=3.432043in,y=0.466921in,,top]{\color{textcolor}{\rmfamily\fontsize{10.000000}{12.000000}\selectfont\catcode`\^=\active\def^{\ifmmode\sp\else\^{}\fi}\catcode`\%=\active\def%{\%}$2^{17}$}}%
\end{pgfscope}%
\begin{pgfscope}%
\pgfsetbuttcap%
\pgfsetroundjoin%
\definecolor{currentfill}{rgb}{0.000000,0.000000,0.000000}%
\pgfsetfillcolor{currentfill}%
\pgfsetlinewidth{0.803000pt}%
\definecolor{currentstroke}{rgb}{0.000000,0.000000,0.000000}%
\pgfsetstrokecolor{currentstroke}%
\pgfsetdash{}{0pt}%
\pgfsys@defobject{currentmarker}{\pgfqpoint{0.000000in}{-0.048611in}}{\pgfqpoint{0.000000in}{0.000000in}}{%
\pgfpathmoveto{\pgfqpoint{0.000000in}{0.000000in}}%
\pgfpathlineto{\pgfqpoint{0.000000in}{-0.048611in}}%
\pgfusepath{stroke,fill}%
}%
\begin{pgfscope}%
\pgfsys@transformshift{3.981892in}{0.564143in}%
\pgfsys@useobject{currentmarker}{}%
\end{pgfscope}%
\end{pgfscope}%
\begin{pgfscope}%
\definecolor{textcolor}{rgb}{0.000000,0.000000,0.000000}%
\pgfsetstrokecolor{textcolor}%
\pgfsetfillcolor{textcolor}%
\pgftext[x=3.981892in,y=0.466921in,,top]{\color{textcolor}{\rmfamily\fontsize{10.000000}{12.000000}\selectfont\catcode`\^=\active\def^{\ifmmode\sp\else\^{}\fi}\catcode`\%=\active\def%{\%}$2^{20}$}}%
\end{pgfscope}%
\begin{pgfscope}%
\pgfsetbuttcap%
\pgfsetroundjoin%
\definecolor{currentfill}{rgb}{0.000000,0.000000,0.000000}%
\pgfsetfillcolor{currentfill}%
\pgfsetlinewidth{0.803000pt}%
\definecolor{currentstroke}{rgb}{0.000000,0.000000,0.000000}%
\pgfsetstrokecolor{currentstroke}%
\pgfsetdash{}{0pt}%
\pgfsys@defobject{currentmarker}{\pgfqpoint{0.000000in}{-0.048611in}}{\pgfqpoint{0.000000in}{0.000000in}}{%
\pgfpathmoveto{\pgfqpoint{0.000000in}{0.000000in}}%
\pgfpathlineto{\pgfqpoint{0.000000in}{-0.048611in}}%
\pgfusepath{stroke,fill}%
}%
\begin{pgfscope}%
\pgfsys@transformshift{4.531740in}{0.564143in}%
\pgfsys@useobject{currentmarker}{}%
\end{pgfscope}%
\end{pgfscope}%
\begin{pgfscope}%
\definecolor{textcolor}{rgb}{0.000000,0.000000,0.000000}%
\pgfsetstrokecolor{textcolor}%
\pgfsetfillcolor{textcolor}%
\pgftext[x=4.531740in,y=0.466921in,,top]{\color{textcolor}{\rmfamily\fontsize{10.000000}{12.000000}\selectfont\catcode`\^=\active\def^{\ifmmode\sp\else\^{}\fi}\catcode`\%=\active\def%{\%}$2^{23}$}}%
\end{pgfscope}%
\begin{pgfscope}%
\pgfsetbuttcap%
\pgfsetroundjoin%
\definecolor{currentfill}{rgb}{0.000000,0.000000,0.000000}%
\pgfsetfillcolor{currentfill}%
\pgfsetlinewidth{0.803000pt}%
\definecolor{currentstroke}{rgb}{0.000000,0.000000,0.000000}%
\pgfsetstrokecolor{currentstroke}%
\pgfsetdash{}{0pt}%
\pgfsys@defobject{currentmarker}{\pgfqpoint{0.000000in}{-0.048611in}}{\pgfqpoint{0.000000in}{0.000000in}}{%
\pgfpathmoveto{\pgfqpoint{0.000000in}{0.000000in}}%
\pgfpathlineto{\pgfqpoint{0.000000in}{-0.048611in}}%
\pgfusepath{stroke,fill}%
}%
\begin{pgfscope}%
\pgfsys@transformshift{5.081589in}{0.564143in}%
\pgfsys@useobject{currentmarker}{}%
\end{pgfscope}%
\end{pgfscope}%
\begin{pgfscope}%
\definecolor{textcolor}{rgb}{0.000000,0.000000,0.000000}%
\pgfsetstrokecolor{textcolor}%
\pgfsetfillcolor{textcolor}%
\pgftext[x=5.081589in,y=0.466921in,,top]{\color{textcolor}{\rmfamily\fontsize{10.000000}{12.000000}\selectfont\catcode`\^=\active\def^{\ifmmode\sp\else\^{}\fi}\catcode`\%=\active\def%{\%}$2^{26}$}}%
\end{pgfscope}%
\begin{pgfscope}%
\definecolor{textcolor}{rgb}{0.000000,0.000000,0.000000}%
\pgfsetstrokecolor{textcolor}%
\pgfsetfillcolor{textcolor}%
\pgftext[x=2.997067in,y=0.287909in,,top]{\color{textcolor}{\rmfamily\fontsize{10.000000}{12.000000}\selectfont\catcode`\^=\active\def^{\ifmmode\sp\else\^{}\fi}\catcode`\%=\active\def%{\%}Array Size (log scale)}}%
\end{pgfscope}%
\begin{pgfscope}%
\pgfsetbuttcap%
\pgfsetroundjoin%
\definecolor{currentfill}{rgb}{0.000000,0.000000,0.000000}%
\pgfsetfillcolor{currentfill}%
\pgfsetlinewidth{0.803000pt}%
\definecolor{currentstroke}{rgb}{0.000000,0.000000,0.000000}%
\pgfsetstrokecolor{currentstroke}%
\pgfsetdash{}{0pt}%
\pgfsys@defobject{currentmarker}{\pgfqpoint{-0.048611in}{0.000000in}}{\pgfqpoint{-0.000000in}{0.000000in}}{%
\pgfpathmoveto{\pgfqpoint{-0.000000in}{0.000000in}}%
\pgfpathlineto{\pgfqpoint{-0.048611in}{0.000000in}}%
\pgfusepath{stroke,fill}%
}%
\begin{pgfscope}%
\pgfsys@transformshift{0.717889in}{0.945283in}%
\pgfsys@useobject{currentmarker}{}%
\end{pgfscope}%
\end{pgfscope}%
\begin{pgfscope}%
\definecolor{textcolor}{rgb}{0.000000,0.000000,0.000000}%
\pgfsetstrokecolor{textcolor}%
\pgfsetfillcolor{textcolor}%
\pgftext[x=0.551222in, y=0.897058in, left, base]{\color{textcolor}{\rmfamily\fontsize{10.000000}{12.000000}\selectfont\catcode`\^=\active\def^{\ifmmode\sp\else\^{}\fi}\catcode`\%=\active\def%{\%}0}}%
\end{pgfscope}%
\begin{pgfscope}%
\pgfsetbuttcap%
\pgfsetroundjoin%
\definecolor{currentfill}{rgb}{0.000000,0.000000,0.000000}%
\pgfsetfillcolor{currentfill}%
\pgfsetlinewidth{0.803000pt}%
\definecolor{currentstroke}{rgb}{0.000000,0.000000,0.000000}%
\pgfsetstrokecolor{currentstroke}%
\pgfsetdash{}{0pt}%
\pgfsys@defobject{currentmarker}{\pgfqpoint{-0.048611in}{0.000000in}}{\pgfqpoint{-0.000000in}{0.000000in}}{%
\pgfpathmoveto{\pgfqpoint{-0.000000in}{0.000000in}}%
\pgfpathlineto{\pgfqpoint{-0.048611in}{0.000000in}}%
\pgfusepath{stroke,fill}%
}%
\begin{pgfscope}%
\pgfsys@transformshift{0.717889in}{1.563273in}%
\pgfsys@useobject{currentmarker}{}%
\end{pgfscope}%
\end{pgfscope}%
\begin{pgfscope}%
\definecolor{textcolor}{rgb}{0.000000,0.000000,0.000000}%
\pgfsetstrokecolor{textcolor}%
\pgfsetfillcolor{textcolor}%
\pgftext[x=0.412332in, y=1.515048in, left, base]{\color{textcolor}{\rmfamily\fontsize{10.000000}{12.000000}\selectfont\catcode`\^=\active\def^{\ifmmode\sp\else\^{}\fi}\catcode`\%=\active\def%{\%}200}}%
\end{pgfscope}%
\begin{pgfscope}%
\pgfsetbuttcap%
\pgfsetroundjoin%
\definecolor{currentfill}{rgb}{0.000000,0.000000,0.000000}%
\pgfsetfillcolor{currentfill}%
\pgfsetlinewidth{0.803000pt}%
\definecolor{currentstroke}{rgb}{0.000000,0.000000,0.000000}%
\pgfsetstrokecolor{currentstroke}%
\pgfsetdash{}{0pt}%
\pgfsys@defobject{currentmarker}{\pgfqpoint{-0.048611in}{0.000000in}}{\pgfqpoint{-0.000000in}{0.000000in}}{%
\pgfpathmoveto{\pgfqpoint{-0.000000in}{0.000000in}}%
\pgfpathlineto{\pgfqpoint{-0.048611in}{0.000000in}}%
\pgfusepath{stroke,fill}%
}%
\begin{pgfscope}%
\pgfsys@transformshift{0.717889in}{2.181263in}%
\pgfsys@useobject{currentmarker}{}%
\end{pgfscope}%
\end{pgfscope}%
\begin{pgfscope}%
\definecolor{textcolor}{rgb}{0.000000,0.000000,0.000000}%
\pgfsetstrokecolor{textcolor}%
\pgfsetfillcolor{textcolor}%
\pgftext[x=0.412332in, y=2.133038in, left, base]{\color{textcolor}{\rmfamily\fontsize{10.000000}{12.000000}\selectfont\catcode`\^=\active\def^{\ifmmode\sp\else\^{}\fi}\catcode`\%=\active\def%{\%}\textnormal{400}}}%
\end{pgfscope}%
\begin{pgfscope}%
\pgfsetbuttcap%
\pgfsetroundjoin%
\definecolor{currentfill}{rgb}{0.000000,0.000000,0.000000}%
\pgfsetfillcolor{currentfill}%
\pgfsetlinewidth{0.803000pt}%
\definecolor{currentstroke}{rgb}{0.000000,0.000000,0.000000}%
\pgfsetstrokecolor{currentstroke}%
\pgfsetdash{}{0pt}%
\pgfsys@defobject{currentmarker}{\pgfqpoint{-0.048611in}{0.000000in}}{\pgfqpoint{-0.000000in}{0.000000in}}{%
\pgfpathmoveto{\pgfqpoint{-0.000000in}{0.000000in}}%
\pgfpathlineto{\pgfqpoint{-0.048611in}{0.000000in}}%
\pgfusepath{stroke,fill}%
}%
\begin{pgfscope}%
\pgfsys@transformshift{0.717889in}{2.799253in}%
\pgfsys@useobject{currentmarker}{}%
\end{pgfscope}%
\end{pgfscope}%
\begin{pgfscope}%
\definecolor{textcolor}{rgb}{0.000000,0.000000,0.000000}%
\pgfsetstrokecolor{textcolor}%
\pgfsetfillcolor{textcolor}%
\pgftext[x=0.412332in, y=2.751028in, left, base]{\color{textcolor}{\rmfamily\fontsize{10.000000}{12.000000}\selectfont\catcode`\^=\active\def^{\ifmmode\sp\else\^{}\fi}\catcode`\%=\active\def%{\%}\textnormal{600}}}%
\end{pgfscope}%
\begin{pgfscope}%
\pgfsetbuttcap%
\pgfsetroundjoin%
\definecolor{currentfill}{rgb}{0.000000,0.000000,0.000000}%
\pgfsetfillcolor{currentfill}%
\pgfsetlinewidth{0.803000pt}%
\definecolor{currentstroke}{rgb}{0.000000,0.000000,0.000000}%
\pgfsetstrokecolor{currentstroke}%
\pgfsetdash{}{0pt}%
\pgfsys@defobject{currentmarker}{\pgfqpoint{-0.048611in}{0.000000in}}{\pgfqpoint{-0.000000in}{0.000000in}}{%
\pgfpathmoveto{\pgfqpoint{-0.000000in}{0.000000in}}%
\pgfpathlineto{\pgfqpoint{-0.048611in}{0.000000in}}%
\pgfusepath{stroke,fill}%
}%
\begin{pgfscope}%
\pgfsys@transformshift{0.717889in}{3.417243in}%
\pgfsys@useobject{currentmarker}{}%
\end{pgfscope}%
\end{pgfscope}%
\begin{pgfscope}%
\definecolor{textcolor}{rgb}{0.000000,0.000000,0.000000}%
\pgfsetstrokecolor{textcolor}%
\pgfsetfillcolor{textcolor}%
\pgftext[x=0.412332in, y=3.369017in, left, base]{\color{textcolor}{\rmfamily\fontsize{10.000000}{12.000000}\selectfont\catcode`\^=\active\def^{\ifmmode\sp\else\^{}\fi}\catcode`\%=\active\def%{\%}$800$}}%
\end{pgfscope}%
\begin{pgfscope}%
\pgfsetbuttcap%
\pgfsetroundjoin%
\definecolor{currentfill}{rgb}{0.000000,0.000000,0.000000}%
\pgfsetfillcolor{currentfill}%
\pgfsetlinewidth{0.803000pt}%
\definecolor{currentstroke}{rgb}{0.000000,0.000000,0.000000}%
\pgfsetstrokecolor{currentstroke}%
\pgfsetdash{}{0pt}%
\pgfsys@defobject{currentmarker}{\pgfqpoint{-0.048611in}{0.000000in}}{\pgfqpoint{-0.000000in}{0.000000in}}{%
\pgfpathmoveto{\pgfqpoint{-0.000000in}{0.000000in}}%
\pgfpathlineto{\pgfqpoint{-0.048611in}{0.000000in}}%
\pgfusepath{stroke,fill}%
}%
\begin{pgfscope}%
\pgfsys@transformshift{0.717889in}{4.035233in}%
\pgfsys@useobject{currentmarker}{}%
\end{pgfscope}%
\end{pgfscope}%
\begin{pgfscope}%
\definecolor{textcolor}{rgb}{0.000000,0.000000,0.000000}%
\pgfsetstrokecolor{textcolor}%
\pgfsetfillcolor{textcolor}%
\pgftext[x=0.342888in, y=3.987007in, left, base]{\color{textcolor}{\rmfamily\fontsize{10.000000}{12.000000}\selectfont\catcode`\^=\active\def^{\ifmmode\sp\else\^{}\fi}\catcode`\%=\active\def%{\%}1000}}%
\end{pgfscope}%
\begin{pgfscope}%
\definecolor{textcolor}{rgb}{0.000000,0.000000,0.000000}%
\pgfsetstrokecolor{textcolor}%
\pgfsetfillcolor{textcolor}%
\pgftext[x=0.287332in,y=2.607072in,,bottom,rotate=90.000000]{\color{textcolor}{\rmfamily\fontsize{10.000000}{12.000000}\selectfont\catcode`\^=\active\def^{\ifmmode\sp\else\^{}\fi}\catcode`\%=\active\def%{\%}Time (ms)}}%
\end{pgfscope}%
\begin{pgfscope}%
\pgfpathrectangle{\pgfqpoint{0.717889in}{0.564143in}}{\pgfqpoint{4.558356in}{4.085857in}}%
\pgfusepath{clip}%
\pgfsetrectcap%
\pgfsetroundjoin%
\pgfsetlinewidth{1.505625pt}%
\definecolor{currentstroke}{rgb}{1.000000,0.000000,0.000000}%
\pgfsetstrokecolor{currentstroke}%
\pgfsetdash{}{0pt}%
\pgfpathmoveto{\pgfqpoint{0.925087in}{0.749864in}}%
\pgfpathlineto{\pgfqpoint{0.966945in}{0.787383in}}%
\pgfpathlineto{\pgfqpoint{1.008803in}{0.824903in}}%
\pgfpathlineto{\pgfqpoint{1.050661in}{0.862422in}}%
\pgfpathlineto{\pgfqpoint{1.092519in}{0.899941in}}%
\pgfpathlineto{\pgfqpoint{1.134378in}{0.937461in}}%
\pgfpathlineto{\pgfqpoint{1.176236in}{0.974980in}}%
\pgfpathlineto{\pgfqpoint{1.218094in}{1.012499in}}%
\pgfpathlineto{\pgfqpoint{1.259952in}{1.050019in}}%
\pgfpathlineto{\pgfqpoint{1.301810in}{1.087538in}}%
\pgfpathlineto{\pgfqpoint{1.343668in}{1.125058in}}%
\pgfpathlineto{\pgfqpoint{1.385527in}{1.162577in}}%
\pgfpathlineto{\pgfqpoint{1.427385in}{1.200096in}}%
\pgfpathlineto{\pgfqpoint{1.469243in}{1.237616in}}%
\pgfpathlineto{\pgfqpoint{1.511101in}{1.275135in}}%
\pgfpathlineto{\pgfqpoint{1.552959in}{1.312654in}}%
\pgfpathlineto{\pgfqpoint{1.594817in}{1.350174in}}%
\pgfpathlineto{\pgfqpoint{1.636676in}{1.387693in}}%
\pgfpathlineto{\pgfqpoint{1.678534in}{1.425212in}}%
\pgfpathlineto{\pgfqpoint{1.720392in}{1.462732in}}%
\pgfpathlineto{\pgfqpoint{1.762250in}{1.500251in}}%
\pgfpathlineto{\pgfqpoint{1.804108in}{1.537770in}}%
\pgfpathlineto{\pgfqpoint{1.845967in}{1.575290in}}%
\pgfpathlineto{\pgfqpoint{1.887825in}{1.612809in}}%
\pgfpathlineto{\pgfqpoint{1.929683in}{1.650328in}}%
\pgfpathlineto{\pgfqpoint{1.971541in}{1.687848in}}%
\pgfpathlineto{\pgfqpoint{2.013399in}{1.725367in}}%
\pgfpathlineto{\pgfqpoint{2.055257in}{1.762886in}}%
\pgfpathlineto{\pgfqpoint{2.097116in}{1.800406in}}%
\pgfpathlineto{\pgfqpoint{2.138974in}{1.837925in}}%
\pgfpathlineto{\pgfqpoint{2.180832in}{1.875444in}}%
\pgfpathlineto{\pgfqpoint{2.222690in}{1.912964in}}%
\pgfpathlineto{\pgfqpoint{2.264548in}{1.950483in}}%
\pgfpathlineto{\pgfqpoint{2.306407in}{1.988002in}}%
\pgfpathlineto{\pgfqpoint{2.348265in}{2.025522in}}%
\pgfpathlineto{\pgfqpoint{2.390123in}{2.063041in}}%
\pgfpathlineto{\pgfqpoint{2.431981in}{2.100560in}}%
\pgfpathlineto{\pgfqpoint{2.473839in}{2.138080in}}%
\pgfpathlineto{\pgfqpoint{2.515697in}{2.175599in}}%
\pgfpathlineto{\pgfqpoint{2.557556in}{2.213119in}}%
\pgfpathlineto{\pgfqpoint{2.599414in}{2.250638in}}%
\pgfpathlineto{\pgfqpoint{2.641272in}{2.288157in}}%
\pgfpathlineto{\pgfqpoint{2.683130in}{2.325677in}}%
\pgfpathlineto{\pgfqpoint{2.724988in}{2.363196in}}%
\pgfpathlineto{\pgfqpoint{2.766847in}{2.400715in}}%
\pgfpathlineto{\pgfqpoint{2.808705in}{2.438235in}}%
\pgfpathlineto{\pgfqpoint{2.850563in}{2.475754in}}%
\pgfpathlineto{\pgfqpoint{2.892421in}{2.513273in}}%
\pgfpathlineto{\pgfqpoint{2.934279in}{2.550793in}}%
\pgfpathlineto{\pgfqpoint{2.976137in}{2.588312in}}%
\pgfpathlineto{\pgfqpoint{3.017996in}{2.625831in}}%
\pgfpathlineto{\pgfqpoint{3.059854in}{2.663351in}}%
\pgfpathlineto{\pgfqpoint{3.101712in}{2.700870in}}%
\pgfpathlineto{\pgfqpoint{3.143570in}{2.738389in}}%
\pgfpathlineto{\pgfqpoint{3.185428in}{2.775909in}}%
\pgfpathlineto{\pgfqpoint{3.227287in}{2.813428in}}%
\pgfpathlineto{\pgfqpoint{3.269145in}{2.850947in}}%
\pgfpathlineto{\pgfqpoint{3.311003in}{2.888467in}}%
\pgfpathlineto{\pgfqpoint{3.352861in}{2.925986in}}%
\pgfpathlineto{\pgfqpoint{3.394719in}{2.963505in}}%
\pgfpathlineto{\pgfqpoint{3.436577in}{3.001025in}}%
\pgfpathlineto{\pgfqpoint{3.478436in}{3.038544in}}%
\pgfpathlineto{\pgfqpoint{3.520294in}{3.076063in}}%
\pgfpathlineto{\pgfqpoint{3.562152in}{3.113583in}}%
\pgfpathlineto{\pgfqpoint{3.604010in}{3.151102in}}%
\pgfpathlineto{\pgfqpoint{3.645868in}{3.188621in}}%
\pgfpathlineto{\pgfqpoint{3.687726in}{3.226141in}}%
\pgfpathlineto{\pgfqpoint{3.729585in}{3.263660in}}%
\pgfpathlineto{\pgfqpoint{3.771443in}{3.301180in}}%
\pgfpathlineto{\pgfqpoint{3.813301in}{3.338699in}}%
\pgfpathlineto{\pgfqpoint{3.855159in}{3.376218in}}%
\pgfpathlineto{\pgfqpoint{3.897017in}{3.413738in}}%
\pgfpathlineto{\pgfqpoint{3.938876in}{3.451257in}}%
\pgfpathlineto{\pgfqpoint{3.980734in}{3.488776in}}%
\pgfpathlineto{\pgfqpoint{4.022592in}{3.526296in}}%
\pgfpathlineto{\pgfqpoint{4.064450in}{3.563815in}}%
\pgfpathlineto{\pgfqpoint{4.106308in}{3.601334in}}%
\pgfpathlineto{\pgfqpoint{4.148166in}{3.638854in}}%
\pgfpathlineto{\pgfqpoint{4.190025in}{3.676373in}}%
\pgfpathlineto{\pgfqpoint{4.231883in}{3.713892in}}%
\pgfpathlineto{\pgfqpoint{4.273741in}{3.751412in}}%
\pgfpathlineto{\pgfqpoint{4.315599in}{3.788931in}}%
\pgfpathlineto{\pgfqpoint{4.357457in}{3.826450in}}%
\pgfpathlineto{\pgfqpoint{4.399316in}{3.863970in}}%
\pgfpathlineto{\pgfqpoint{4.441174in}{3.901489in}}%
\pgfpathlineto{\pgfqpoint{4.483032in}{3.939008in}}%
\pgfpathlineto{\pgfqpoint{4.524890in}{3.976528in}}%
\pgfpathlineto{\pgfqpoint{4.566748in}{4.014047in}}%
\pgfpathlineto{\pgfqpoint{4.608606in}{4.051566in}}%
\pgfpathlineto{\pgfqpoint{4.650465in}{4.089086in}}%
\pgfpathlineto{\pgfqpoint{4.692323in}{4.126605in}}%
\pgfpathlineto{\pgfqpoint{4.734181in}{4.164124in}}%
\pgfpathlineto{\pgfqpoint{4.776039in}{4.201644in}}%
\pgfpathlineto{\pgfqpoint{4.817897in}{4.239163in}}%
\pgfpathlineto{\pgfqpoint{4.859756in}{4.276683in}}%
\pgfpathlineto{\pgfqpoint{4.901614in}{4.314202in}}%
\pgfpathlineto{\pgfqpoint{4.943472in}{4.351721in}}%
\pgfpathlineto{\pgfqpoint{4.985330in}{4.389241in}}%
\pgfpathlineto{\pgfqpoint{5.027188in}{4.426760in}}%
\pgfpathlineto{\pgfqpoint{5.069046in}{4.464279in}}%
\pgfusepath{stroke}%
\end{pgfscope}%
\begin{pgfscope}%
\pgfsetrectcap%
\pgfsetmiterjoin%
\pgfsetlinewidth{0.803000pt}%
\definecolor{currentstroke}{rgb}{0.000000,0.000000,0.000000}%
\pgfsetstrokecolor{currentstroke}%
\pgfsetdash{}{0pt}%
\pgfpathmoveto{\pgfqpoint{0.717889in}{0.564143in}}%
\pgfpathlineto{\pgfqpoint{0.717889in}{4.650000in}}%
\pgfusepath{stroke}%
\end{pgfscope}%
\begin{pgfscope}%
\pgfsetrectcap%
\pgfsetmiterjoin%
\pgfsetlinewidth{0.803000pt}%
\definecolor{currentstroke}{rgb}{0.000000,0.000000,0.000000}%
\pgfsetstrokecolor{currentstroke}%
\pgfsetdash{}{0pt}%
\pgfpathmoveto{\pgfqpoint{5.276244in}{0.564143in}}%
\pgfpathlineto{\pgfqpoint{5.276244in}{4.650000in}}%
\pgfusepath{stroke}%
\end{pgfscope}%
\begin{pgfscope}%
\pgfsetrectcap%
\pgfsetmiterjoin%
\pgfsetlinewidth{0.803000pt}%
\definecolor{currentstroke}{rgb}{0.000000,0.000000,0.000000}%
\pgfsetstrokecolor{currentstroke}%
\pgfsetdash{}{0pt}%
\pgfpathmoveto{\pgfqpoint{0.717889in}{0.564143in}}%
\pgfpathlineto{\pgfqpoint{5.276244in}{0.564143in}}%
\pgfusepath{stroke}%
\end{pgfscope}%
\begin{pgfscope}%
\pgfsetrectcap%
\pgfsetmiterjoin%
\pgfsetlinewidth{0.803000pt}%
\definecolor{currentstroke}{rgb}{0.000000,0.000000,0.000000}%
\pgfsetstrokecolor{currentstroke}%
\pgfsetdash{}{0pt}%
\pgfpathmoveto{\pgfqpoint{0.717889in}{4.650000in}}%
\pgfpathlineto{\pgfqpoint{5.276244in}{4.650000in}}%
\pgfusepath{stroke}%
\end{pgfscope}%
\begin{pgfscope}%
\definecolor{textcolor}{rgb}{0.000000,0.501961,0.000000}%
\pgfsetstrokecolor{textcolor}%
\pgfsetfillcolor{textcolor}%
\pgftext[x=5.069046in,y=4.464279in,left,bottom]{\color{textcolor}{\rmfamily\fontsize{10.000000}{12.000000}\selectfont\catcode`\^=\active\def^{\ifmmode\sp\else\^{}\fi}\catcode`\%=\active\def%{\%}  (64000000, 1138.85)}}%
\end{pgfscope}%
\begin{pgfscope}%
\pgfpathrectangle{\pgfqpoint{0.717889in}{0.564143in}}{\pgfqpoint{4.558356in}{4.085857in}}%
\pgfusepath{clip}%
\pgfsetbuttcap%
\pgfsetroundjoin%
\definecolor{currentfill}{rgb}{0.000000,0.501961,0.000000}%
\pgfsetfillcolor{currentfill}%
\pgfsetlinewidth{1.003750pt}%
\definecolor{currentstroke}{rgb}{0.000000,0.501961,0.000000}%
\pgfsetstrokecolor{currentstroke}%
\pgfsetdash{}{0pt}%
\pgfsys@defobject{currentmarker}{\pgfqpoint{-0.041667in}{-0.041667in}}{\pgfqpoint{0.041667in}{0.041667in}}{%
\pgfpathmoveto{\pgfqpoint{0.000000in}{-0.041667in}}%
\pgfpathcurveto{\pgfqpoint{0.011050in}{-0.041667in}}{\pgfqpoint{0.021649in}{-0.037276in}}{\pgfqpoint{0.029463in}{-0.029463in}}%
\pgfpathcurveto{\pgfqpoint{0.037276in}{-0.021649in}}{\pgfqpoint{0.041667in}{-0.011050in}}{\pgfqpoint{0.041667in}{0.000000in}}%
\pgfpathcurveto{\pgfqpoint{0.041667in}{0.011050in}}{\pgfqpoint{0.037276in}{0.021649in}}{\pgfqpoint{0.029463in}{0.029463in}}%
\pgfpathcurveto{\pgfqpoint{0.021649in}{0.037276in}}{\pgfqpoint{0.011050in}{0.041667in}}{\pgfqpoint{0.000000in}{0.041667in}}%
\pgfpathcurveto{\pgfqpoint{-0.011050in}{0.041667in}}{\pgfqpoint{-0.021649in}{0.037276in}}{\pgfqpoint{-0.029463in}{0.029463in}}%
\pgfpathcurveto{\pgfqpoint{-0.037276in}{0.021649in}}{\pgfqpoint{-0.041667in}{0.011050in}}{\pgfqpoint{-0.041667in}{0.000000in}}%
\pgfpathcurveto{\pgfqpoint{-0.041667in}{-0.011050in}}{\pgfqpoint{-0.037276in}{-0.021649in}}{\pgfqpoint{-0.029463in}{-0.029463in}}%
\pgfpathcurveto{\pgfqpoint{-0.021649in}{-0.037276in}}{\pgfqpoint{-0.011050in}{-0.041667in}}{\pgfqpoint{0.000000in}{-0.041667in}}%
\pgfpathlineto{\pgfqpoint{0.000000in}{-0.041667in}}%
\pgfpathclose%
\pgfusepath{stroke,fill}%
}%
\begin{pgfscope}%
\pgfsys@transformshift{5.069046in}{4.464279in}%
\pgfsys@useobject{currentmarker}{}%
\end{pgfscope}%
\end{pgfscope}%
\begin{pgfscope}%
\pgfsetbuttcap%
\pgfsetmiterjoin%
\definecolor{currentfill}{rgb}{1.000000,1.000000,1.000000}%
\pgfsetfillcolor{currentfill}%
\pgfsetfillopacity{0.800000}%
\pgfsetlinewidth{1.003750pt}%
\definecolor{currentstroke}{rgb}{0.800000,0.800000,0.800000}%
\pgfsetstrokecolor{currentstroke}%
\pgfsetstrokeopacity{0.800000}%
\pgfsetdash{}{0pt}%
\pgfpathmoveto{\pgfqpoint{0.815111in}{4.151543in}}%
\pgfpathlineto{\pgfqpoint{2.214418in}{4.151543in}}%
\pgfpathquadraticcurveto{\pgfqpoint{2.242195in}{4.151543in}}{\pgfqpoint{2.242195in}{4.179321in}}%
\pgfpathlineto{\pgfqpoint{2.242195in}{4.552778in}}%
\pgfpathquadraticcurveto{\pgfqpoint{2.242195in}{4.580556in}}{\pgfqpoint{2.214418in}{4.580556in}}%
\pgfpathlineto{\pgfqpoint{0.815111in}{4.580556in}}%
\pgfpathquadraticcurveto{\pgfqpoint{0.787333in}{4.580556in}}{\pgfqpoint{0.787333in}{4.552778in}}%
\pgfpathlineto{\pgfqpoint{0.787333in}{4.179321in}}%
\pgfpathquadraticcurveto{\pgfqpoint{0.787333in}{4.151543in}}{\pgfqpoint{0.815111in}{4.151543in}}%
\pgfpathlineto{\pgfqpoint{0.815111in}{4.151543in}}%
\pgfpathclose%
\pgfusepath{stroke,fill}%
\end{pgfscope}%
\begin{pgfscope}%
\pgfsetbuttcap%
\pgfsetroundjoin%
\definecolor{currentfill}{rgb}{0.000000,0.000000,1.000000}%
\pgfsetfillcolor{currentfill}%
\pgfsetlinewidth{1.003750pt}%
\definecolor{currentstroke}{rgb}{0.000000,0.000000,1.000000}%
\pgfsetstrokecolor{currentstroke}%
\pgfsetdash{}{0pt}%
\pgfsys@defobject{currentmarker}{\pgfqpoint{-0.041667in}{-0.041667in}}{\pgfqpoint{0.041667in}{0.041667in}}{%
\pgfpathmoveto{\pgfqpoint{0.000000in}{-0.041667in}}%
\pgfpathcurveto{\pgfqpoint{0.011050in}{-0.041667in}}{\pgfqpoint{0.021649in}{-0.037276in}}{\pgfqpoint{0.029463in}{-0.029463in}}%
\pgfpathcurveto{\pgfqpoint{0.037276in}{-0.021649in}}{\pgfqpoint{0.041667in}{-0.011050in}}{\pgfqpoint{0.041667in}{0.000000in}}%
\pgfpathcurveto{\pgfqpoint{0.041667in}{0.011050in}}{\pgfqpoint{0.037276in}{0.021649in}}{\pgfqpoint{0.029463in}{0.029463in}}%
\pgfpathcurveto{\pgfqpoint{0.021649in}{0.037276in}}{\pgfqpoint{0.011050in}{0.041667in}}{\pgfqpoint{0.000000in}{0.041667in}}%
\pgfpathcurveto{\pgfqpoint{-0.011050in}{0.041667in}}{\pgfqpoint{-0.021649in}{0.037276in}}{\pgfqpoint{-0.029463in}{0.029463in}}%
\pgfpathcurveto{\pgfqpoint{-0.037276in}{0.021649in}}{\pgfqpoint{-0.041667in}{0.011050in}}{\pgfqpoint{-0.041667in}{0.000000in}}%
\pgfpathcurveto{\pgfqpoint{-0.041667in}{-0.011050in}}{\pgfqpoint{-0.037276in}{-0.021649in}}{\pgfqpoint{-0.029463in}{-0.029463in}}%
\pgfpathcurveto{\pgfqpoint{-0.021649in}{-0.037276in}}{\pgfqpoint{-0.011050in}{-0.041667in}}{\pgfqpoint{0.000000in}{-0.041667in}}%
\pgfpathlineto{\pgfqpoint{0.000000in}{-0.041667in}}%
\pgfpathclose%
\pgfusepath{stroke,fill}%
}%
\begin{pgfscope}%
\pgfsys@transformshift{0.981778in}{4.464236in}%
\pgfsys@useobject{currentmarker}{}%
\end{pgfscope}%
\end{pgfscope}%
\begin{pgfscope}%
\definecolor{textcolor}{rgb}{0.000000,0.000000,0.000000}%
\pgfsetstrokecolor{textcolor}%
\pgfsetfillcolor{textcolor}%
\pgftext[x=1.231778in,y=4.427778in,left,base]{\color{textcolor}{\rmfamily\fontsize{10.000000}{12.000000}\selectfont\catcode`\^=\active\def^{\ifmmode\sp\else\^{}\fi}\catcode`\%=\active\def%{\%}Data}}%
\end{pgfscope}%
\begin{pgfscope}%
\pgfsetrectcap%
\pgfsetroundjoin%
\pgfsetlinewidth{1.505625pt}%
\definecolor{currentstroke}{rgb}{1.000000,0.000000,0.000000}%
\pgfsetstrokecolor{currentstroke}%
\pgfsetdash{}{0pt}%
\pgfpathmoveto{\pgfqpoint{0.842889in}{4.282716in}}%
\pgfpathlineto{\pgfqpoint{0.981778in}{4.282716in}}%
\pgfpathlineto{\pgfqpoint{1.120666in}{4.282716in}}%
\pgfusepath{stroke}%
\end{pgfscope}%
\begin{pgfscope}%
\definecolor{textcolor}{rgb}{0.000000,0.000000,0.000000}%
\pgfsetstrokecolor{textcolor}%
\pgfsetfillcolor{textcolor}%
\pgftext[x=1.231778in,y=4.234105in,left,base]{\color{textcolor}{\rmfamily\fontsize{10.000000}{12.000000}\selectfont\catcode`\^=\active\def^{\ifmmode\sp\else\^{}\fi}\catcode`\%=\active\def%{\%}Regression Line}}%
\end{pgfscope}%
\end{pgfpicture}%
\makeatother%
\endgroup%

  }
  \caption{Regression line for binary search}
  \label{fig:regression}
\end{figure}

\section*{Recursive Binary Search}

The following code implements the recursive binary search algorithm.
The difference between the original binary search and the recursive binary search is that
the recursive binary search calls itself to search the left or right side of the
array while the original binary search uses a while loop to search the array.

\begin{minted}{c}
bool recursive(int array[], int length, int key, int first, int last) {
  int index = (first + last) / 2; // jump to the middle
  if (array[index] == key) {
    return true;
  } else if (array[index] < key && index < last) {
    first = index + 1;
    recursive(array, length, key, first, last);
  } else if (array[index] > key && index > first) {
    last = index - 1;
    recursive(array, length, key, first, last);
  } else {
    return false;
  }
}
\end{minted}

The time taken to perform recursive binary search is shown in the Figure \ref{fig:recursive_search},
which is similar to the binary search.

\begin{figure}[H]
  \centering
  \begin{tikzpicture}
  \begin{axis}[
    xlabel={Array Size (per element)},
    ylabel={Time (ms)},
    xmode=log,
    log basis x=10,
    ymajorgrids=true,
    grid style=dashed,
    width=12cm, height=6cm,
    legend pos=outer north east,
    scaled y ticks=false,
    scaled x ticks=false,
    % y filter/.code={\pgfmathparse{#1/1000000}\pgfmathresult},
    % x filter/.code={\pgfmathparse{#1/1000000}\pgfmathresult},
    ]
  \addplot[color=blue,] table {data/recursive.dat};
  \end{axis}

  \end{tikzpicture}
  \caption{Time taken to search using regression binary search}
  \label{fig:recursive_search}
\end{figure}

Though the recursive binary search is more readable and easier to implement, 
it uses more memory then the original one, because when the function calls itself,
it leaves the current function in the stack and enters a new function.

When it is n levels deep, we have n stack frames in the memory.
The memory complexity of the recursive binary 
search is proportional to the depth of the recursion,  
which is $O(\log n)$ while the original binary search is $O(1)$.

The call times of the recursive binary search is shown in the Figure \ref{fig:recursive_call},
which is proportional to the logarithm of the size of the array.

\begin{figure}[H]
  \centering
  \begin{tikzpicture}
  \begin{axis}[
    xlabel={Array Size (per element)},
    ylabel={Call Times},
    xmode=log,
    log basis x=10,
    ymajorgrids=true,
    grid style=dashed,
    width=12cm, height=6cm,
    legend pos=outer north east,
    scaled y ticks=false,
    scaled x ticks=false,
    ]
  \addplot[color=blue,] table {data/recursive_call_times.dat};
  \end{axis}

  \end{tikzpicture}
  \caption{Call times of the recursive binary search}
  \label{fig:recursive_call}
\end{figure}

\end{document}
