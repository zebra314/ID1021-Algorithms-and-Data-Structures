\documentclass[a4paper,11pt]{article}

\usepackage[utf8]{inputenc}

\usepackage{graphicx}
\usepackage{caption}
\usepackage{subcaption}
\usepackage{float}
\usepackage{amsmath}

\usepackage{pgfplots}
\pgfplotsset{compat=1.18} 

\usepackage{minted}

\begin{document}

\title{
    \textbf{A Calculator in C}
}
\author{Ying Pei Lin}
\date{Fall 2024}

\maketitle

\section*{Static stack}

To implement a static stack in C, we can use an array to store the elements and an integer to record
the current size of the stack. The following code shows how I initialize the stack. 

\begin{minted}{c}
stack *new_stack(int size) {
  int *array = (int*)malloc(size*sizeof(int));
  stack *stk = (stack*)malloc(sizeof(stack));
  stk->top = 0; // Point to the above the stack.
  stk->size = size;
  stk->array = array;
  return stk;
}
\end{minted}

The top integer points to the next avaialble index in the stack. For expample, if the top is 
at index 3, this means that the stack has 3 elements stored at index 0, 1, and 2. When the stack
is empty, the top equals to 0. And the following code shows how to push and pop elements from
the stack.

\begin{minted}{c}
void push(stack *stk, int val) {
  if(stk->top >= (stk->size)) {
    printf("Overflow ");
  } else {
    stk->array[stk->top] = val;
    stk->top++;
  }
}

int pop(stack *stk) {
  if(stk->top == 0) {
    printf("Underflow ");
    return INT_MIN;
  } else {
    stk->top--;
    return stk->array[stk->top];
  }
}
\end{minted}

We check for overflow when pushing an element and check for underflow when popping an element.
Since C doesn't support excpetion handling, we need to check the stack size before pushing 
and poping. If the stack is empty, we return {\tt INT\_MIN} to indicate that the stack is empty and
if the stack is full, we just need to print out the error message.

\section*{Dynamic stack}

To implement a dynamic stack, we need to resize the stack when it's full. I made this modification
in the push function. First, I check if the stack is full. If it is, then create a new array 
with the double size and copy the data from the old one to the new one. At the end, free the old
array and set old array to the new one. By adding the following code to the push function, we can
implement the dynamic stack.

\begin{minted}{c}
if (stk->top == stk->size) {
  int size = stk->size * 2;
  int *copy = (int*)malloc(size*sizeof(int));
  stk->size = size;
  for (int i = 0; i < stk->top; i++) {
    copy[i] = stk->array[i];
  }
  free(stk->array);
  stk->array = copy;
}
\end{minted}

Unless there is a huge amount of unused memory, I decide not to shrink the stack size because 
resizing and copying the data is a costly operation. For this case, I just let the stack grow.

If we want to shrink the stack size, we can modify the push function to check if there is
a lot of unused memory. If the unused memory is more than a certain threshold (e.g. 75\%),
then we can shrink the stack size by half. By adding the following code to the push function,
we can shrink the stack size, which is similar we did in the push function.

\begin{minted}{c}
if (stk->top < stk->size / 4) {
  int size = stk->size / 2;
  int *copy = (int*)malloc(size*sizeof(int));
  stk->size = size;
  for (int i = 0; i < stk->top; i++) {
    copy[i] = stk->array[i];
  }
  free(stk->array);
  stk->array = copy;
}
\end{minted}

\section*{The Calculator}

The HP35 calculator uses reverse Polish notation to perform the calculation. We take the input from the user
and store the numbers in the stack. If the input is a
number then push it to the stack. If it's an operator, then pop two numbers from the stack, and
perform the operation.

For example, if the input is {\tt 4 2 3 * 4 + 4 * + 2 -}, then the stack will look like the following table.

\begin{table}[h!]
  \centering
  \begin{tabular}{|c|p{4cm}|}
  \hline
  \textbf{Input} & \textbf{Stack} \\ 
                 & \textbf{(Top marked with \textbf{*})} \\ \hline
  4      & 4\textbf{*} \\ \hline
  2      & 4 2\textbf{*} \\ \hline
  3      & 4 2 3\textbf{*} \\ \hline
  *      & 4 6\textbf{*} \\ \hline
  4      & 4 6 4\textbf{*} \\ \hline
  +      & 4 10\textbf{*} \\ \hline
  4      & 4 10 4\textbf{*} \\ \hline
  *      & 4 40\textbf{*} \\ \hline
  +      & 44\textbf{*} \\ \hline
  2      & 44 2\textbf{*} \\ \hline
  -      & 42\textbf{*} \\ \hline
  \end{tabular}
  \caption{HP-35 Calculator Steps (Top element marked)}
\end{table}

\end{document}
