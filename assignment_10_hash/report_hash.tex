\documentclass[a4paper,11pt]{article}

\usepackage[utf8]{inputenc}

\usepackage{graphicx}
\usepackage{caption}
\usepackage{subcaption}
\usepackage{float}
\usepackage{amsmath}
\usepackage{array}
\usepackage{multirow}
\usepackage[colorlinks=true, linkcolor=blue, urlcolor=blue]{hyperref}

\usepackage{pgfplots}
\pgfplotsset{compat=1.18} 

\usepackage{minted}
\usepackage{pgfplotstable}

\begin{document}

\title{
  \textbf{Trees in C}
}
\author{Ying Pei Lin}
\date{Fall 2024}

\maketitle

\subsection*{Zip Code Table}

At {\tt read\_postcodes} function, we store the zip codes in the format of char and after
we use the {\tt strcmp} function to compare the zip codes in the search function. Down below
is the code snippet of the search function.

\begin{minted}{c}
area* linear_search_char(codes *postnr, const char *zip) {
  for (int i = 0; i < postnr->n; i++) {
    if (strcmp(postnr->areas[i].zip_char, zip) == 0) {
      return &postnr->areas[i];
    }
  }
  return NULL;
}

area* binary_search_char(codes *postnr, const char *zip) {
  int left = 0;
  int right = postnr->n - 1;
  
  while (left <= right) {
    int mid = (left + right) / 2;
    int cmp = strcmp(postnr->areas[mid].zip_char, zip);
    
    if (cmp == 0) {
      return &postnr->areas[mid];
    }
    if (cmp < 0) {
      left = mid + 1;
    } else {
      right = mid - 1;
    }
  }
  return NULL;
}
\end{minted}

This is not efficient because the {\tt strcmp} function compares the zip codes character by character.
We can improve the search function by converting the zip codes to integers and compare them directly.
The Table \ref{table:linear_times} shows the time taken for linear search with different data types
while Table \ref{table:binary_times} shows the time taken for binary search with different data types.

\begin{table}[h!]
  \centering
  \begin{tabular}{|c|c|c|}
    \hline
    \textbf{ZIP Code} & \textbf{Search Type} & \textbf{Time (ns)} \\ \hline
    "111 15" & Linear Search & 3 ns \\ \hline
    "111 15" & Binary Search & 39 ns \\ \hline
    "984 99" & Linear Search & 25037 ns \\ \hline
    "984 99" & Binary Search & 36 ns \\ \hline
  \end{tabular}
  \caption{Comparison the Search Times for Char Data Type ZIP Codes in Linear and Binary Searches}
  \label{table:linear_times}
\end{table}

\begin{table}[h!]
  \centering
  \begin{tabular}{|c|c|c|}
    \hline
    \textbf{ZIP Code} & \textbf{Search Type} & \textbf{Time (ns)} \\ \hline
    111 15 & Linear Search & 0 ns \\ \hline
    111 15 & Binary Search & 18 ns \\ \hline
    984 99 & Linear Search & 5076 ns \\ \hline
    984 99 & Binary Search & 32 ns \\ \hline
  \end{tabular}
  \caption{Comparison the Search Times for Integer Data Type ZIP Codes in Linear and Binary Searches}
  \label{table:binary_times}
\end{table}

Before comparing the results, we should note that these two cases are the edge cases. 
The first case 111 15 is the first element in the array, and the second case 984 99 is the last element in the array.

For the first case, the linear search is faster than the binary search because the linear search can find the element in the first iteration.
For the second case, the binary search is faster than the linear search because it does not have to iterate through all the elements to find the last element.
Additionally, the data type does affect the search time. The integer data type is faster than the character data type in both linear and binary searches.

\subsection*{Direct Indexing}

The direct indexing method is a way to improve the search time by using the zip code as an index to the array.
To achieve this, I add a new {\tt areas\_direct} array is an array of pointers to the {\tt area} struct
and use the zip code as the index to the array.
The modified {\tt codes} struct is as follows:

\begin{minted}{c}
typedef struct codes {
  area *areas;
  area **areas_direct; // Array of pointers to areas
  int n;
} codes;
\end{minted}

To initialize the direct indexing, I create a new function {\tt init\_direct}
to convert the zip code to an integer and use it as the index to the array after store the post codes
using the original {\tt read\_postcodes} function.

\begin{minted}{c}
codes *init_direct(codes *postnr) {
  postnr->areas_direct = (area**)malloc(sizeof(area*)*100000);
  for(int i = 0; i < postnr->n; i++) {
    int zip = zip_to_int(postnr->areas[i].zip_char);
    postnr->areas_direct[zip] = &postnr->areas[i];
  }
  return postnr;
}
\end{minted}

Comparing the search time for the direct indexing method with the binary search methods,
the direct indexing method is faster than the binary search method as shown in Table \ref{table:direct_times}.
The direct indexing method is faster because it uses the zip code as an index to the array, which is a constant time operation.

\begin{table}[h!]
  \centering
  \begin{tabular}{|c|c|c|}
    \hline
    \textbf{ZIP Code} & \textbf{Search Type} & \textbf{Time (ns)} \\ \hline
    111 15 & Direct lookup & 0 ns \\ \hline
    111 15 & Binary Search & 18 ns \\ \hline
    984 99 & Direct lookup & 0 ns \\ \hline
    984 99 & Binary Search & 20 ns \\ \hline
  \end{tabular}
  \caption{Comparison the Search Times for Integer Data Type ZIP Codes in Direct lookup and Binary Searches}
  \label{table:direct_times}
\end{table}

\subsection*{Collisions}

The direct indexing method is efficient when there are no collisions. However, when there are collisions,
it will overwrite the previous element in the array. To handle collisions, we can use a linked list to store the elements with the same index.
Here, I use a array of pointers to the {\tt node} struct to store the elements with the same index.
The method is called bucket hashing.

I modified the {\tt codes} struct to include the {\tt buckets} array 
of pointers to the {\tt node} struct. The modified {\tt codes} struct is as follows:

\begin{minted}{c}
typedef struct bucket {
  area *areas;        // Dynamic array of areas
  int size;           // Current size of bucket
  int capacity;       // Capacity of bucket
} bucket;

typedef struct codes {
  bucket **buckets;    // Array of pointers to buckets
  int n;               // Number of areas
  int size;            // Size of hash table
} codes;
\end{minted}

To insert the elements into the hash table, I use the {\tt hash} function to convert the zip code to an integer
and use it as the index to the array. No matter if the collision happens or not, the element will be added to the bucket.
If the bucket is full, I resize the bucket by doubling the capacity.

\begin{minted}{c}
int hash(int zip, int table_size) {
  return zip % table_size;
}

void insert_bucket(codes *postnr, area *a) {
  int index = hash(a->zip_int, postnr->size);
  
  // If bucket doesn't exist, create it
  if(postnr->buckets[index] == NULL) {
    postnr->buckets[index] = malloc(sizeof(bucket));
    postnr->buckets[index]->capacity = 1;
    postnr->buckets[index]->size = 0;
    postnr->buckets[index]->areas = malloc(sizeof(area));
  }
  
  // If bucket is full, resize it
  bucket *b = postnr->buckets[index];
  if(b->size >= b->capacity) {
    b->capacity *= 2;
    b->areas = realloc(b->areas, b->capacity * sizeof(area));
  }
  
  // Add the area
  b->areas[b->size] = *a;
  b->size++;
}
\end{minted}

The search function is similar to the direct indexing method, 
but it has to iterate through the bucket to find the element.

\begin{minted}{c}
area* hash_lookup(codes *postnr, int zip) {
  int index = hash(zip, postnr->size);
  bucket *b = postnr->buckets[index];
  
  for(int i = 0; i < b->size; i++) {
    if(b->areas[i].zip_int == zip) {
      return &b->areas[i];
    }
  }
  return NULL;
}
\end{minted}

The efficiency of the bucket hashing method depends on the size of the hash table and the number of collisions.
If the hash table is too small, there will be many collisions, and the search time will increase.
If the hash table is too large, there will be many empty buckets, and the memory usage will increase.
Comparing the amount of collisions between different size of hash table, the Table \ref{table:collisions} shows 
the number of collisions for different size of hash table(measured using the modified {\tt collisions} function).

\begin{table}[h!]
  \centering
  \begin{tabular}{|c|c|c|c|c|c|c|}
    \hline
    \multirow{2}{*}{\textbf{Size of Hash Table}} & \multicolumn{6}{c|}{\textbf{Numbers of Collision}} \\ 
    \cline{2-7} & \textbf{0} & \textbf{1} & \textbf{2} & \textbf{3} & \textbf{4} & \textbf{5} \\ \hline
    13513 & 5410 & 1678 & 287 & 12 & 0 & 0 \\ \hline
    13600 & 3406 & 1578 & 613 & 229 & 56 & 13 \\ \hline
    14000 & 3055 & 1316 & 615 & 320 & 134 & 31 \\ \hline
  \end{tabular}
  \caption{Comparison of the Number of Collisions for Different Sizes of Hash Table}
  \label{table:collisions}
\end{table}

\subsection*{}


\end{document}
