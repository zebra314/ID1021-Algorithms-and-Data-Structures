\documentclass[a4paper,11pt]{article}

\usepackage[utf8]{inputenc}

\usepackage{graphicx}
\usepackage{caption}
\usepackage{subcaption}
\usepackage{float}
\usepackage{amsmath}
\usepackage{array}
\usepackage[colorlinks=true, linkcolor=blue, urlcolor=blue]{hyperref}

\usepackage{pgfplots}
\pgfplotsset{compat=1.18} 

\usepackage{minted}
\usepackage{pgfplotstable}

\begin{document}

\title{
  \textbf{Trees in C}
}
\author{Ying Pei Lin}
\date{Fall 2024}

\maketitle

\subsection*{Zip Code Table}

At {\tt read\_postcodes} function, we store the zip codes in the format of char and after
we use the {\tt strcmp} function to compare the zip codes in the search function. Down below
is the code snippet of the search function.

\begin{minted}{c}
area* linear_search_char(codes *postnr, const char *zip) {
  for (int i = 0; i < postnr->n; i++) {
    if (strcmp(postnr->areas[i].zip_char, zip) == 0) {
      return &postnr->areas[i];
    }
  }
  return NULL;
}

area* binary_search_char(codes *postnr, const char *zip) {
  int left = 0;
  int right = postnr->n - 1;
  
  while (left <= right) {
    int mid = (left + right) / 2;
    int cmp = strcmp(postnr->areas[mid].zip_char, zip);
    
    if (cmp == 0) {
      return &postnr->areas[mid];
    }
    if (cmp < 0) {
      left = mid + 1;
    } else {
      right = mid - 1;
    }
  }
  return NULL;
}
\end{minted}

This is not efficient because the {\tt strcmp} function compares the zip codes character by character.
We can improve the search function by converting the zip codes to integers and compare them directly.
The Table \ref{table:linear_times} shows the time taken for linear search with different data types
while Table \ref{table:binary_times} shows the time taken for binary search with different data types.

\begin{table}[h!]
  \centering
  \begin{tabular}{|c|c|c|}
    \hline
    \textbf{ZIP Code} & \textbf{Search Type} & \textbf{Time (ns)} \\ \hline
    "111 15" & Linear Search & 3 ns \\ \hline
    "111 15" & Binary Search & 39 ns \\ \hline
    "984 99" & Linear Search & 25037 ns \\ \hline
    "984 99" & Binary Search & 36 ns \\ \hline
  \end{tabular}
  \caption{Comparison the Search Times for Char Data Type ZIP Codes in Linear and Binary Searches}
  \label{table:linear_times}
\end{table}

\begin{table}[h!]
  \centering
  \begin{tabular}{|c|c|c|}
    \hline
    \textbf{ZIP Code} & \textbf{Search Type} & \textbf{Time (ns)} \\ \hline
    111 15 & Linear Search & 0 ns \\ \hline
    111 15 & Binary Search & 18 ns \\ \hline
    984 99 & Linear Search & 5076 ns \\ \hline
    984 99 & Binary Search & 32 ns \\ \hline
  \end{tabular}
  \caption{Comparison the Search Times for Integer Data Type ZIP Codes in Linear and Binary Searches}
  \label{table:binary_times}
\end{table}

Before comparing the results, we should note that these two cases are the edge cases. 
The first case 111 15 is the first element in the array, and the second case 984 99 is the last element in the array.

For the first case, the linear search is faster than the binary search because the linear search can find the element in the first iteration.
For the second case, the binary search is faster than the linear search because it does not have to iterate through all the elements to find the last element.
Additionally, the data type does affect the search time. The integer data type is faster than the character data type in both linear and binary searches.

\subsection*{Direct Indexing}

\end{document}
